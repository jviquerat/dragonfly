%%%%%%%%%%%%%%%%%%%%%%%%%%%%%%%%%%%%%%%%%%%%%%%%%%%%%%%%%%
%%%%%%%%%%%%%%%%%%%%%%%%%%%%%%%%%%%%%%%%%%%%%%%%%%%%%%%%%%
%%%%%%%%%%%%%%%%%%%%%%%%%%%%%%%%%%%%%%%%%%%%%%%%%%%%%%%%%%
%%%%%%%%%%%%%%%%%%%%%%%%%%%%%%%%%%%%%%%%%%%%%%%%%%%%%%%%%%
\chapter{Installation and usage}

%%%%%%%%%%%%%%%%%%%%%%%%%%%%%%%%%%%%%%%%%%%%%%%%%%%%%%%%%%
%%%%%%%%%%%%%%%%%%%%%%%%%%%%%%%%%%%%%%%%%%%%%%%%%%%%%%%%%%
%%%%%%%%%%%%%%%%%%%%%%%%%%%%%%%%%%%%%%%%%%%%%%%%%%%%%%%%%%
\section{Requirements}

As of today, the library is tested against python \codeinline{3.8.7} on ubuntu \codeinline{20.04} and osx \codeinline{12.5}. The main required packages versions are the following:

\begin{minted}{python}
tensorflow==2.9.1
gym==0.26.0
mujoco==2.2.0
numpy==1.22.4
\end{minted}

The full list of required packages can be found in \codeinline{dragonfly/requirements.txt}.

%%%%%%%%%%%%%%%%%%%%%%%%%%%%%%%%%%%%%%%%%%%%%%%%%%%%%%%%%%
%%%%%%%%%%%%%%%%%%%%%%%%%%%%%%%%%%%%%%%%%%%%%%%%%%%%%%%%%%
%%%%%%%%%%%%%%%%%%%%%%%%%%%%%%%%%%%%%%%%%%%%%%%%%%%%%%%%%%
\section{Installation}

Clone the repository locally, then install it using \codeinline{pip}:

\begin{minted}{python}
git clone git@github.com:jviquerat/dragonfly.git
cd dragonfly
pip install -e .
\end{minted}

Once installed, the different tasks can be run directly from the command line through the \codeinline{dgf} executable (see next section).

%%%%%%%%%%%%%%%%%%%%%%%%%%%%%%%%%%%%%%%%%%%%%%%%%%%%%%%%%%
%%%%%%%%%%%%%%%%%%%%%%%%%%%%%%%%%%%%%%%%%%%%%%%%%%%%%%%%%%
%%%%%%%%%%%%%%%%%%%%%%%%%%%%%%%%%%%%%%%%%%%%%%%%%%%%%%%%%%
\section{Usage}

%%%%%%%%%%%%%%%%%%%%%%%%%%%%%%%%%%%%%%%%%%%%%%%%%%%%%%%%%%
%%%%%%%%%%%%%%%%%%%%%%%%%%%%%%%%%%%%%%%%%%%%%%%%%%%%%%%%%%
\subsection{Training mode}

The training mode relies on the use of a \codeinline{.json} file for the description of the agent and its training mode (see section \textcolor{red}{to complete} for a complete description of the json parameter files):

\begin{minted}{python}
dgf --train <filename>.json
\end{minted}

The completion of a training phase will produce a \codeinline{results/} repository in the current folder, in which training data and saved models will be stored (see section \textcolor{red}{to complete} for a description of the saved data). It will also produce a plot of the average and standard deviation of the score against training steps (see example in figure \ref{fig:training_score}).

%%%%%%%%%%%%
%%%%%%%%%%%%
\begin{figure}
\centering
%%%%%%%%%%%%
\begin{tikzpicture}[	trim axis left, trim axis right, font=\scriptsize,
				upper/.style={	name path=upper, smooth, draw=none},
				lower/.style={	name path=lower, smooth, draw=none},]
	\begin{axis}[	xmin=0, xmax=1000000, scale=0.75,
				ymin=-1000, ymax=6000,
				scaled x ticks=false,
				xtick={0,250000,500000,750000,1000000},
				xticklabels={$0$,$250k$,$500k$,$750k$,$1000k$},
				ytick={-1000,0,2000,4000,6000},
				yticklabels={$-1k$,$0$,$2k$,$4k$,$6k$},
				legend cell align=left, legend pos=south east,
				legend style={nodes={scale=0.8, transform shape}},
				every tick label/.append style={font=\scriptsize},
				grid=major, xlabel=transitions, ylabel=score]
				
		\legend{\sac}
		
		\addplot [upper, forget plot] 				table[x index=0,y index=7] {fig/mujoco/ant/sac.dat};
		\addplot [lower, forget plot] 				table[x index=0,y index=6] {fig/mujoco/ant/sac.dat}; 
		\addplot [fill=blue3, opacity=0.5, forget plot] 	fill between[of=upper and lower];
		\addplot[draw=blue1, thick, smooth] 			table[x index=0,y index=5] {fig/mujoco/ant/sac.dat}; 
			
	\end{axis}
\end{tikzpicture}
%%%%%%%%%%%%
\caption{\textbf{Example of score report outputted after training} of the \sac algorithm on the \codeinline{ant-v4} environment from the \codeinline{mujoco} package.} 
\label{fig:training_score}
\end{figure} 
%%%%%%%%%%%%
%%%%%%%%%%%%

%%%%%%%%%%%%%%%%%%%%%%%%%%%%%%%%%%%%%%%%%%%%%%%%%%%%%%%%%%
%%%%%%%%%%%%%%%%%%%%%%%%%%%%%%%%%%%%%%%%%%%%%%%%%%%%%%%%%%
\subsection{Exploitation mode}

The exploitation mode requires the same \codeinline{.json} file used for training, as well as a model file. Model files are saved regularly during the training phase, and their saving/loading rely on the tensorflow \codeinline{save_weights} and \codeinline{load_weights} interfaces. The syntax to exploit a saved model is the following:

\begin{minted}{python}
dgf --eval -net <path/to/network/model> -json <filename>.json 
\end{minted}

In that case, the environment will just rely on the \codeinline{done} signal to stop the evaluation. Alternatively, you can provide a \codeinline{-steps <n_steps>} option, that will override the done signal of the environment, and force its execution for \codeinline{n_steps} steps:

\begin{minted}{python}
dgf --eval -net <path/to/network/model> -json <filename>.json -steps <n_steps>
\end{minted}

It is also possible to add a warmup phase, during which default control values are passed to the environment:

\begin{minted}{python}
dgf --eval -net ... -warmup <n_steps_warmup> <default_control_values>
\end{minted}