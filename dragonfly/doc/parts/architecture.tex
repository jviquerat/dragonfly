%%%%%%%%%%%%%%%%%%%%%%%%%%%%%%%%%%%%%%%%%%%%%%%%%%%%%%%%%%
%%%%%%%%%%%%%%%%%%%%%%%%%%%%%%%%%%%%%%%%%%%%%%%%%%%%%%%%%%
%%%%%%%%%%%%%%%%%%%%%%%%%%%%%%%%%%%%%%%%%%%%%%%%%%%%%%%%%%
%%%%%%%%%%%%%%%%%%%%%%%%%%%%%%%%%%%%%%%%%%%%%%%%%%%%%%%%%%
\chapter{Library architecture}
\label{section:architecture}

%%%%%%%%%%%%%%%%%%%%%%%%%%%%%%%%%%%%%%%%%%%%%%%%%%%%%%%%%%%
%%%%%%%%%%%%%%%%%%%%%%%%%%%%%%%%%%%%%%%%%%%%%%%%%%%%%%%%%%%
%%%%%%%%%%%%%%%%%%%%%%%%%%%%%%%%%%%%%%%%%%%%%%%%%%%%%%%%%%%
\section{Factory pattern}
\label{section:factory}

The library is mostly based on a factory pattern, which is used to return objects from given classes from a string representation:

\begin{minted}{python}
class factory:
    def __init__(self):
        self.keys = {}

    def register(self, key, creator):
        self.keys[key] = creator

    def create(self, key, **kwargs):
        creator = self.keys.get(key)
        if not creator:
            try:
                raise ValueError(key)
            except ValueError:
                error("factory", "create", "Unknown key provided: "+key)
                raise
        return creator(**kwargs)
\end{minted}

For each type of object (agents, losses, etc), a factory is instantiated, and the different corresponding classes are registered using the \codeinline{register} member:

\begin{minted}{python}
agent_factory = factory()

agent_factory.register("a2c",  a2c)
agent_factory.register("ppo",  ppo)
agent_factory.register("dqn",  dqn)
agent_factory.register("ddpg", ddpg)
agent_factory.register("td3",  td3)
agent_factory.register("sac",  sac)
\end{minted}

Then, generating an object of a given type (here an agent) from a string representation is a fairly simple task (additional parameters for the constructor of the object can also be passed):

\begin{minted}{python}
self.agent = agent_factory.create(agent_pms.type,
                                  obs_dim = self.obs_dim,
                                  act_dim = self.act_dim,
                                  n_cpu   = self.n_cpu,
                                  size    = self.size,
                                  pms     = agent_pms)
\end{minted}

This design allows to modify an existing algorithm by simply replacing a building block by another directly in the \codeinline{.json} configuration file. Examples are given in section \textcolor{red}{to complete}.

%%%%%%%%%%%%%%%%%%%%%%%%%%%%%%%%%%%%%%%%%%%%%%%%%%%%%%%%%%%
%%%%%%%%%%%%%%%%%%%%%%%%%%%%%%%%%%%%%%%%%%%%%%%%%%%%%%%%%%%
%%%%%%%%%%%%%%%%%%%%%%%%%%%%%%%%%%%%%%%%%%%%%%%%%%%%%%%%%%%
\section{General overview}
\label{section:general_overview}

The global architecture of the library is summed up below:

%%%%%%%%%%%%%%%%%%%%%%%%%%%%%%%%%%%%%%%%%%%%%%%%%%%%%%%%%%
%%%%%%%%%%%%%%%%%%%%%%%%%%%%%%%%%%%%%%%%%%%%%%%%%%%%%%%%%%
%%%%%%%%%%%%%%%%%%%%%%%%%%%%%%%%%%%%%%%%%%%%%%%%%%%%%%%%%%
%%%%%%%%%%%%%%%%%%%%%%%%%%%%%%%%%%%%%%%%%%%%%%%%%%%%%%%%%%
\chapter{Library architecture}
\label{section:architecture}

%%%%%%%%%%%%%%%%%%%%%%%%%%%%%%%%%%%%%%%%%%%%%%%%%%%%%%%%%%%
%%%%%%%%%%%%%%%%%%%%%%%%%%%%%%%%%%%%%%%%%%%%%%%%%%%%%%%%%%%
%%%%%%%%%%%%%%%%%%%%%%%%%%%%%%%%%%%%%%%%%%%%%%%%%%%%%%%%%%%
\section{Factory pattern}
\label{section:factory}

The library is mostly based on a factory pattern, which is used to return objects from given classes from a string representation:

\begin{minted}{python}
class factory:
    def __init__(self):
        self.keys = {}

    def register(self, key, creator):
        self.keys[key] = creator

    def create(self, key, **kwargs):
        creator = self.keys.get(key)
        if not creator:
            try:
                raise ValueError(key)
            except ValueError:
                error("factory", "create", "Unknown key provided: "+key)
                raise
        return creator(**kwargs)
\end{minted}

For each type of object (agents, losses, etc), a factory is instantiated, and the different corresponding classes are registered using the \codeinline{register} member:

\begin{minted}{python}
agent_factory = factory()

agent_factory.register("a2c",  a2c)
agent_factory.register("ppo",  ppo)
agent_factory.register("dqn",  dqn)
agent_factory.register("ddpg", ddpg)
agent_factory.register("td3",  td3)
agent_factory.register("sac",  sac)
\end{minted}

Then, generating an object of a given type (here an agent) from a string representation is a fairly simple task (additional parameters for the constructor of the object can also be passed):

\begin{minted}{python}
self.agent = agent_factory.create(agent_pms.type,
                                  obs_dim = self.obs_dim,
                                  act_dim = self.act_dim,
                                  n_cpu   = self.n_cpu,
                                  size    = self.size,
                                  pms     = agent_pms)
\end{minted}

This design allows to modify an existing algorithm by simply replacing a building block by another directly in the \codeinline{.json} configuration file. Examples are given in section \textcolor{red}{to complete}.

%%%%%%%%%%%%%%%%%%%%%%%%%%%%%%%%%%%%%%%%%%%%%%%%%%%%%%%%%%%
%%%%%%%%%%%%%%%%%%%%%%%%%%%%%%%%%%%%%%%%%%%%%%%%%%%%%%%%%%%
%%%%%%%%%%%%%%%%%%%%%%%%%%%%%%%%%%%%%%%%%%%%%%%%%%%%%%%%%%%
\section{General overview}
\label{section:general_overview}

The global architecture of the library is summed up below:

%%%%%%%%%%%%%%%%%%%%%%%%%%%%%%%%%%%%%%%%%%%%%%%%%%%%%%%%%%
%%%%%%%%%%%%%%%%%%%%%%%%%%%%%%%%%%%%%%%%%%%%%%%%%%%%%%%%%%
%%%%%%%%%%%%%%%%%%%%%%%%%%%%%%%%%%%%%%%%%%%%%%%%%%%%%%%%%%
%%%%%%%%%%%%%%%%%%%%%%%%%%%%%%%%%%%%%%%%%%%%%%%%%%%%%%%%%%
\chapter{Library architecture}
\label{section:architecture}

%%%%%%%%%%%%%%%%%%%%%%%%%%%%%%%%%%%%%%%%%%%%%%%%%%%%%%%%%%%
%%%%%%%%%%%%%%%%%%%%%%%%%%%%%%%%%%%%%%%%%%%%%%%%%%%%%%%%%%%
%%%%%%%%%%%%%%%%%%%%%%%%%%%%%%%%%%%%%%%%%%%%%%%%%%%%%%%%%%%
\section{Factory pattern}
\label{section:factory}

The library is mostly based on a factory pattern, which is used to return objects from given classes from a string representation:

\begin{minted}{python}
class factory:
    def __init__(self):
        self.keys = {}

    def register(self, key, creator):
        self.keys[key] = creator

    def create(self, key, **kwargs):
        creator = self.keys.get(key)
        if not creator:
            try:
                raise ValueError(key)
            except ValueError:
                error("factory", "create", "Unknown key provided: "+key)
                raise
        return creator(**kwargs)
\end{minted}

For each type of object (agents, losses, etc), a factory is instantiated, and the different corresponding classes are registered using the \codeinline{register} member:

\begin{minted}{python}
agent_factory = factory()

agent_factory.register("a2c",  a2c)
agent_factory.register("ppo",  ppo)
agent_factory.register("dqn",  dqn)
agent_factory.register("ddpg", ddpg)
agent_factory.register("td3",  td3)
agent_factory.register("sac",  sac)
\end{minted}

Then, generating an object of a given type (here an agent) from a string representation is a fairly simple task (additional parameters for the constructor of the object can also be passed):

\begin{minted}{python}
self.agent = agent_factory.create(agent_pms.type,
                                  obs_dim = self.obs_dim,
                                  act_dim = self.act_dim,
                                  n_cpu   = self.n_cpu,
                                  size    = self.size,
                                  pms     = agent_pms)
\end{minted}

This design allows to modify an existing algorithm by simply replacing a building block by another directly in the \codeinline{.json} configuration file. Examples are given in section \textcolor{red}{to complete}.

%%%%%%%%%%%%%%%%%%%%%%%%%%%%%%%%%%%%%%%%%%%%%%%%%%%%%%%%%%%
%%%%%%%%%%%%%%%%%%%%%%%%%%%%%%%%%%%%%%%%%%%%%%%%%%%%%%%%%%%
%%%%%%%%%%%%%%%%%%%%%%%%%%%%%%%%%%%%%%%%%%%%%%%%%%%%%%%%%%%
\section{General overview}
\label{section:general_overview}

The global architecture of the library is summed up below:

%%%%%%%%%%%%%%%%%%%%%%%%%%%%%%%%%%%%%%%%%%%%%%%%%%%%%%%%%%
%%%%%%%%%%%%%%%%%%%%%%%%%%%%%%%%%%%%%%%%%%%%%%%%%%%%%%%%%%
%%%%%%%%%%%%%%%%%%%%%%%%%%%%%%%%%%%%%%%%%%%%%%%%%%%%%%%%%%
%%%%%%%%%%%%%%%%%%%%%%%%%%%%%%%%%%%%%%%%%%%%%%%%%%%%%%%%%%
\chapter{Library architecture}
\label{section:architecture}

%%%%%%%%%%%%%%%%%%%%%%%%%%%%%%%%%%%%%%%%%%%%%%%%%%%%%%%%%%%
%%%%%%%%%%%%%%%%%%%%%%%%%%%%%%%%%%%%%%%%%%%%%%%%%%%%%%%%%%%
%%%%%%%%%%%%%%%%%%%%%%%%%%%%%%%%%%%%%%%%%%%%%%%%%%%%%%%%%%%
\section{Factory pattern}
\label{section:factory}

The library is mostly based on a factory pattern, which is used to return objects from given classes from a string representation:

\begin{minted}{python}
class factory:
    def __init__(self):
        self.keys = {}

    def register(self, key, creator):
        self.keys[key] = creator

    def create(self, key, **kwargs):
        creator = self.keys.get(key)
        if not creator:
            try:
                raise ValueError(key)
            except ValueError:
                error("factory", "create", "Unknown key provided: "+key)
                raise
        return creator(**kwargs)
\end{minted}

For each type of object (agents, losses, etc), a factory is instantiated, and the different corresponding classes are registered using the \codeinline{register} member:

\begin{minted}{python}
agent_factory = factory()

agent_factory.register("a2c",  a2c)
agent_factory.register("ppo",  ppo)
agent_factory.register("dqn",  dqn)
agent_factory.register("ddpg", ddpg)
agent_factory.register("td3",  td3)
agent_factory.register("sac",  sac)
\end{minted}

Then, generating an object of a given type (here an agent) from a string representation is a fairly simple task (additional parameters for the constructor of the object can also be passed):

\begin{minted}{python}
self.agent = agent_factory.create(agent_pms.type,
                                  obs_dim = self.obs_dim,
                                  act_dim = self.act_dim,
                                  n_cpu   = self.n_cpu,
                                  size    = self.size,
                                  pms     = agent_pms)
\end{minted}

This design allows to modify an existing algorithm by simply replacing a building block by another directly in the \codeinline{.json} configuration file. Examples are given in section \textcolor{red}{to complete}.

%%%%%%%%%%%%%%%%%%%%%%%%%%%%%%%%%%%%%%%%%%%%%%%%%%%%%%%%%%%
%%%%%%%%%%%%%%%%%%%%%%%%%%%%%%%%%%%%%%%%%%%%%%%%%%%%%%%%%%%
%%%%%%%%%%%%%%%%%%%%%%%%%%%%%%%%%%%%%%%%%%%%%%%%%%%%%%%%%%%
\section{General overview}
\label{section:general_overview}

The global architecture of the library is summed up below:

\input{fig/architecture}

The training is driven by a \codeinline{.json} file that provides all the informations required to set the different classes and train the agent. The training procedure is handled by a \codeinline{trainer} object, which in turn initializes the \codeinline{environment} and the \codeinline{agent}, as well as several other objects helpful for the logging, rendering, etc. The \codeinline{environment} class can then spin up different parallel versions of itself to accelerate training (this part is handled using the \codeinline{multiprocessing} library), while the agent generates the instances that are required for its training, such as policies, values and buffers. These elements rely on a lower level of objects that are networks, optimizers and losses. The whole training procedure can be automatically performed several times to obtain averaged quantities of interest.


The training is driven by a \codeinline{.json} file that provides all the informations required to set the different classes and train the agent. The training procedure is handled by a \codeinline{trainer} object, which in turn initializes the \codeinline{environment} and the \codeinline{agent}, as well as several other objects helpful for the logging, rendering, etc. The \codeinline{environment} class can then spin up different parallel versions of itself to accelerate training (this part is handled using the \codeinline{multiprocessing} library), while the agent generates the instances that are required for its training, such as policies, values and buffers. These elements rely on a lower level of objects that are networks, optimizers and losses. The whole training procedure can be automatically performed several times to obtain averaged quantities of interest.


The training is driven by a \codeinline{.json} file that provides all the informations required to set the different classes and train the agent. The training procedure is handled by a \codeinline{trainer} object, which in turn initializes the \codeinline{environment} and the \codeinline{agent}, as well as several other objects helpful for the logging, rendering, etc. The \codeinline{environment} class can then spin up different parallel versions of itself to accelerate training (this part is handled using the \codeinline{multiprocessing} library), while the agent generates the instances that are required for its training, such as policies, values and buffers. These elements rely on a lower level of objects that are networks, optimizers and losses. The whole training procedure can be automatically performed several times to obtain averaged quantities of interest.


The training is driven by a \codeinline{.json} file that provides all the informations required to set the different classes and train the agent. The training procedure is handled by a \codeinline{trainer} object, which in turn initializes the \codeinline{environment} and the \codeinline{agent}, as well as several other objects helpful for the logging, rendering, etc. The \codeinline{environment} class can then spin up different parallel versions of itself to accelerate training (this part is handled using the \codeinline{multiprocessing} library), while the agent generates the instances that are required for its training, such as policies, values and buffers. These elements rely on a lower level of objects that are networks, optimizers and losses. The whole training procedure can be automatically performed several times to obtain averaged quantities of interest.
