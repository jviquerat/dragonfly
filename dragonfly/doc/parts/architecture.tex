%%%%%%%%%%%%%%%%%%%%%%%%%%%%%%%%%%%%%%%%%%%%%%%%%%%%%%%%%%
%%%%%%%%%%%%%%%%%%%%%%%%%%%%%%%%%%%%%%%%%%%%%%%%%%%%%%%%%%
%%%%%%%%%%%%%%%%%%%%%%%%%%%%%%%%%%%%%%%%%%%%%%%%%%%%%%%%%%
%%%%%%%%%%%%%%%%%%%%%%%%%%%%%%%%%%%%%%%%%%%%%%%%%%%%%%%%%%
\chapter{Library architecture}
\label{section:architecture}

%%%%%%%%%%%%%%%%%%%%%%%%%%%%%%%%%%%%%%%%%%%%%%%%%%%%%%%%%%%
%%%%%%%%%%%%%%%%%%%%%%%%%%%%%%%%%%%%%%%%%%%%%%%%%%%%%%%%%%%
%%%%%%%%%%%%%%%%%%%%%%%%%%%%%%%%%%%%%%%%%%%%%%%%%%%%%%%%%%%
\section{Factory pattern}
\label{section:factory}

The library is mostly based on a factory pattern, which is used to return objects from given classes from a string representation:

\begin{minted}{python}
class factory:
    def __init__(self):
        self.keys = {}

    def register(self, key, creator):
        self.keys[key] = creator

    def create(self, key, **kwargs):
        creator = self.keys.get(key)
        if not creator:
            try:
                raise ValueError(key)
            except ValueError:
                error("factory", "create", "Unknown key provided: "+key)
                raise
        return creator(**kwargs)
\end{minted}

For each type of object (agents, losses, etc), a factory is instantiated, and the different corresponding classes are registered using the \codeinline{register} member:

\begin{minted}{python}
agent_factory = factory()

agent_factory.register("a2c",  a2c)
agent_factory.register("ppo",  ppo)
agent_factory.register("dqn",  dqn)
agent_factory.register("ddpg", ddpg)
agent_factory.register("td3",  td3)
agent_factory.register("sac",  sac)
\end{minted}

Then, generating an object of a given type (here an agent) from a string representation is a fairly simple task (additional parameters for the constructor of the object can also be passed):

\begin{minted}{python}
self.agent = agent_factory.create(agent_pms.type,
                                  obs_dim = self.obs_dim,
                                  act_dim = self.act_dim,
                                  n_cpu   = self.n_cpu,
                                  size    = self.size,
                                  pms     = agent_pms)
\end{minted}

This design allows to modify an existing algorithm by simply replacing a building block by another directly in the \codeinline{.json} configuration file. Examples are given in section \textcolor{red}{to complete}.

%%%%%%%%%%%%%%%%%%%%%%%%%%%%%%%%%%%%%%%%%%%%%%%%%%%%%%%%%%%
%%%%%%%%%%%%%%%%%%%%%%%%%%%%%%%%%%%%%%%%%%%%%%%%%%%%%%%%%%%
%%%%%%%%%%%%%%%%%%%%%%%%%%%%%%%%%%%%%%%%%%%%%%%%%%%%%%%%%%%
\section{General overview}
\label{section:general_overview}

The global architecture of the library is summed up in figure \ref{fig:architecture}. The training is driven by a \codeinline{.json} file that provides all the informations required to set the different classes and train the agent. The training procedure is handled by a \codeinline{trainer} object, which in turn initializes the \codeinline{environment} and the \codeinline{agent}, as well as several other objects helpful for the logging, rendering, etc. The \codeinline{environment} class can then spin up different parallel versions of itself to accelerate training (this part is handled using the \codeinline{multiprocessing} library), while the agent generates the instances that are required for its training, such as policies, values and buffers. These elements rely on a lower level of objects that are networks, optimizers and losses. The whole training procedure can be automatically performed several times to obtain averaged quantities of interest.

%%%%%%%%%%%%%%%%%%%%%%%%%%%%%%%%%%%%%%%%%%%%%%%%%%%%%%%%%%
%%%%%%%%%%%%%%%%%%%%%%%%%%%%%%%%%%%%%%%%%%%%%%%%%%%%%%%%%%
%%%%%%%%%%%%%%%%%%%%%%%%%%%%%%%%%%%%%%%%%%%%%%%%%%%%%%%%%%
%%%%%%%%%%%%%%%%%%%%%%%%%%%%%%%%%%%%%%%%%%%%%%%%%%%%%%%%%%
\chapter{Library architecture}
\label{section:architecture}

%%%%%%%%%%%%%%%%%%%%%%%%%%%%%%%%%%%%%%%%%%%%%%%%%%%%%%%%%%%
%%%%%%%%%%%%%%%%%%%%%%%%%%%%%%%%%%%%%%%%%%%%%%%%%%%%%%%%%%%
%%%%%%%%%%%%%%%%%%%%%%%%%%%%%%%%%%%%%%%%%%%%%%%%%%%%%%%%%%%
\section{Factory pattern}
\label{section:factory}

The library is mostly based on a factory pattern, which is used to return objects from given classes from a string representation:

\begin{minted}{python}
class factory:
    def __init__(self):
        self.keys = {}

    def register(self, key, creator):
        self.keys[key] = creator

    def create(self, key, **kwargs):
        creator = self.keys.get(key)
        if not creator:
            try:
                raise ValueError(key)
            except ValueError:
                error("factory", "create", "Unknown key provided: "+key)
                raise
        return creator(**kwargs)
\end{minted}

For each type of object (agents, losses, etc), a factory is instantiated, and the different corresponding classes are registered using the \codeinline{register} member:

\begin{minted}{python}
agent_factory = factory()

agent_factory.register("a2c",  a2c)
agent_factory.register("ppo",  ppo)
agent_factory.register("dqn",  dqn)
agent_factory.register("ddpg", ddpg)
agent_factory.register("td3",  td3)
agent_factory.register("sac",  sac)
\end{minted}

Then, generating an object of a given type (here an agent) from a string representation is a fairly simple task (additional parameters for the constructor of the object can also be passed):

\begin{minted}{python}
self.agent = agent_factory.create(agent_pms.type,
                                  obs_dim = self.obs_dim,
                                  act_dim = self.act_dim,
                                  n_cpu   = self.n_cpu,
                                  size    = self.size,
                                  pms     = agent_pms)
\end{minted}

This design allows to modify an existing algorithm by simply replacing a building block by another directly in the \codeinline{.json} configuration file. Examples are given in section \textcolor{red}{to complete}.

%%%%%%%%%%%%%%%%%%%%%%%%%%%%%%%%%%%%%%%%%%%%%%%%%%%%%%%%%%%
%%%%%%%%%%%%%%%%%%%%%%%%%%%%%%%%%%%%%%%%%%%%%%%%%%%%%%%%%%%
%%%%%%%%%%%%%%%%%%%%%%%%%%%%%%%%%%%%%%%%%%%%%%%%%%%%%%%%%%%
\section{General overview}
\label{section:general_overview}

The global architecture of the library is summed up below:

%%%%%%%%%%%%%%%%%%%%%%%%%%%%%%%%%%%%%%%%%%%%%%%%%%%%%%%%%%
%%%%%%%%%%%%%%%%%%%%%%%%%%%%%%%%%%%%%%%%%%%%%%%%%%%%%%%%%%
%%%%%%%%%%%%%%%%%%%%%%%%%%%%%%%%%%%%%%%%%%%%%%%%%%%%%%%%%%
%%%%%%%%%%%%%%%%%%%%%%%%%%%%%%%%%%%%%%%%%%%%%%%%%%%%%%%%%%
\chapter{Library architecture}
\label{section:architecture}

%%%%%%%%%%%%%%%%%%%%%%%%%%%%%%%%%%%%%%%%%%%%%%%%%%%%%%%%%%%
%%%%%%%%%%%%%%%%%%%%%%%%%%%%%%%%%%%%%%%%%%%%%%%%%%%%%%%%%%%
%%%%%%%%%%%%%%%%%%%%%%%%%%%%%%%%%%%%%%%%%%%%%%%%%%%%%%%%%%%
\section{Factory pattern}
\label{section:factory}

The library is mostly based on a factory pattern, which is used to return objects from given classes from a string representation:

\begin{minted}{python}
class factory:
    def __init__(self):
        self.keys = {}

    def register(self, key, creator):
        self.keys[key] = creator

    def create(self, key, **kwargs):
        creator = self.keys.get(key)
        if not creator:
            try:
                raise ValueError(key)
            except ValueError:
                error("factory", "create", "Unknown key provided: "+key)
                raise
        return creator(**kwargs)
\end{minted}

For each type of object (agents, losses, etc), a factory is instantiated, and the different corresponding classes are registered using the \codeinline{register} member:

\begin{minted}{python}
agent_factory = factory()

agent_factory.register("a2c",  a2c)
agent_factory.register("ppo",  ppo)
agent_factory.register("dqn",  dqn)
agent_factory.register("ddpg", ddpg)
agent_factory.register("td3",  td3)
agent_factory.register("sac",  sac)
\end{minted}

Then, generating an object of a given type (here an agent) from a string representation is a fairly simple task (additional parameters for the constructor of the object can also be passed):

\begin{minted}{python}
self.agent = agent_factory.create(agent_pms.type,
                                  obs_dim = self.obs_dim,
                                  act_dim = self.act_dim,
                                  n_cpu   = self.n_cpu,
                                  size    = self.size,
                                  pms     = agent_pms)
\end{minted}

This design allows to modify an existing algorithm by simply replacing a building block by another directly in the \codeinline{.json} configuration file. Examples are given in section \textcolor{red}{to complete}.

%%%%%%%%%%%%%%%%%%%%%%%%%%%%%%%%%%%%%%%%%%%%%%%%%%%%%%%%%%%
%%%%%%%%%%%%%%%%%%%%%%%%%%%%%%%%%%%%%%%%%%%%%%%%%%%%%%%%%%%
%%%%%%%%%%%%%%%%%%%%%%%%%%%%%%%%%%%%%%%%%%%%%%%%%%%%%%%%%%%
\section{General overview}
\label{section:general_overview}

The global architecture of the library is summed up below:

%%%%%%%%%%%%%%%%%%%%%%%%%%%%%%%%%%%%%%%%%%%%%%%%%%%%%%%%%%
%%%%%%%%%%%%%%%%%%%%%%%%%%%%%%%%%%%%%%%%%%%%%%%%%%%%%%%%%%
%%%%%%%%%%%%%%%%%%%%%%%%%%%%%%%%%%%%%%%%%%%%%%%%%%%%%%%%%%
%%%%%%%%%%%%%%%%%%%%%%%%%%%%%%%%%%%%%%%%%%%%%%%%%%%%%%%%%%
\chapter{Library architecture}
\label{section:architecture}

%%%%%%%%%%%%%%%%%%%%%%%%%%%%%%%%%%%%%%%%%%%%%%%%%%%%%%%%%%%
%%%%%%%%%%%%%%%%%%%%%%%%%%%%%%%%%%%%%%%%%%%%%%%%%%%%%%%%%%%
%%%%%%%%%%%%%%%%%%%%%%%%%%%%%%%%%%%%%%%%%%%%%%%%%%%%%%%%%%%
\section{Factory pattern}
\label{section:factory}

The library is mostly based on a factory pattern, which is used to return objects from given classes from a string representation:

\begin{minted}{python}
class factory:
    def __init__(self):
        self.keys = {}

    def register(self, key, creator):
        self.keys[key] = creator

    def create(self, key, **kwargs):
        creator = self.keys.get(key)
        if not creator:
            try:
                raise ValueError(key)
            except ValueError:
                error("factory", "create", "Unknown key provided: "+key)
                raise
        return creator(**kwargs)
\end{minted}

For each type of object (agents, losses, etc), a factory is instantiated, and the different corresponding classes are registered using the \codeinline{register} member:

\begin{minted}{python}
agent_factory = factory()

agent_factory.register("a2c",  a2c)
agent_factory.register("ppo",  ppo)
agent_factory.register("dqn",  dqn)
agent_factory.register("ddpg", ddpg)
agent_factory.register("td3",  td3)
agent_factory.register("sac",  sac)
\end{minted}

Then, generating an object of a given type (here an agent) from a string representation is a fairly simple task (additional parameters for the constructor of the object can also be passed):

\begin{minted}{python}
self.agent = agent_factory.create(agent_pms.type,
                                  obs_dim = self.obs_dim,
                                  act_dim = self.act_dim,
                                  n_cpu   = self.n_cpu,
                                  size    = self.size,
                                  pms     = agent_pms)
\end{minted}

This design allows to modify an existing algorithm by simply replacing a building block by another directly in the \codeinline{.json} configuration file. Examples are given in section \textcolor{red}{to complete}.

%%%%%%%%%%%%%%%%%%%%%%%%%%%%%%%%%%%%%%%%%%%%%%%%%%%%%%%%%%%
%%%%%%%%%%%%%%%%%%%%%%%%%%%%%%%%%%%%%%%%%%%%%%%%%%%%%%%%%%%
%%%%%%%%%%%%%%%%%%%%%%%%%%%%%%%%%%%%%%%%%%%%%%%%%%%%%%%%%%%
\section{General overview}
\label{section:general_overview}

The global architecture of the library is summed up below:

\input{fig/architecture}

The training is driven by a \codeinline{.json} file that provides all the informations required to set the different classes and train the agent. The training procedure is handled by a \codeinline{trainer} object, which in turn initializes the \codeinline{environment} and the \codeinline{agent}, as well as several other objects helpful for the logging, rendering, etc. The \codeinline{environment} class can then spin up different parallel versions of itself to accelerate training (this part is handled using the \codeinline{multiprocessing} library), while the agent generates the instances that are required for its training, such as policies, values and buffers. These elements rely on a lower level of objects that are networks, optimizers and losses. The whole training procedure can be automatically performed several times to obtain averaged quantities of interest.


The training is driven by a \codeinline{.json} file that provides all the informations required to set the different classes and train the agent. The training procedure is handled by a \codeinline{trainer} object, which in turn initializes the \codeinline{environment} and the \codeinline{agent}, as well as several other objects helpful for the logging, rendering, etc. The \codeinline{environment} class can then spin up different parallel versions of itself to accelerate training (this part is handled using the \codeinline{multiprocessing} library), while the agent generates the instances that are required for its training, such as policies, values and buffers. These elements rely on a lower level of objects that are networks, optimizers and losses. The whole training procedure can be automatically performed several times to obtain averaged quantities of interest.


The training is driven by a \codeinline{.json} file that provides all the informations required to set the different classes and train the agent. The training procedure is handled by a \codeinline{trainer} object, which in turn initializes the \codeinline{environment} and the \codeinline{agent}, as well as several other objects helpful for the logging, rendering, etc. The \codeinline{environment} class can then spin up different parallel versions of itself to accelerate training (this part is handled using the \codeinline{multiprocessing} library), while the agent generates the instances that are required for its training, such as policies, values and buffers. These elements rely on a lower level of objects that are networks, optimizers and losses. The whole training procedure can be automatically performed several times to obtain averaged quantities of interest.


%%%%%%%%%%%%%%%%%%%%%%%%%%%%%%%%%%%%%%%%%%%%%%%%%%%%%%%%%%%
%%%%%%%%%%%%%%%%%%%%%%%%%%%%%%%%%%%%%%%%%%%%%%%%%%%%%%%%%%%
%%%%%%%%%%%%%%%%%%%%%%%%%%%%%%%%%%%%%%%%%%%%%%%%%%%%%%%%%%%
\section{Parameter files}
\label{section:parameter_files}

Parameters for training are provided under the \codeinline{.json} format to the library, using the following general architecture:

\begin{minted}[fontsize=\scriptsize]{json}
{
	"env": {...},
	"naming": {...},
	"n_avg": 5,
	"n_cpu": 1,
	"n_stp_max": 100000,
	"trainer": {...},
	"agent": {...}
}
\end{minted}

Please note that most of the following parameters don't have default values and must therefore be provided in the parameter file.

%%%%%%%%%%%%%%%%%%%%%%%%%%%%%%%%%%%%%%%%%%%%%%%%%%%%%%%%%%%
%%%%%%%%%%%%%%%%%%%%%%%%%%%%%%%%%%%%%%%%%%%%%%%%%%%%%%%%%%%
\subsection{Generic parameters}
\label{section:generic_parameters}

The following parameters must be provided regardless of the chosen trainer, environment or agent:

\begin{enumerate}
	\item The \codeinline{n_avg} key indicates the number of times the full training of the agent will be performed. This is used to obtain averaged training scores, as shown previously in figure \ref{fig:training_score};
	\item The \codeinline{n_cpu} key corresponds to the number of parallel environments that will be used to collect samples. \warning{Depending on the chosen \codeinline{trainer}, this option will have different behaviors:} please refer to section \textcolor{red}{link to trainer section};
	\item The \codeinline{n_stp_max} key indicates the maximal number of steps that will be performed overall, regardless of the number of parallel environments. Please note that running environments will \emph{not} be interrupted if the \codeinline{n_stp_max} is exceeded: the trainer will wait until the next training stage to terminate.
\end{enumerate}

%%%%%%%%%%%%%%%%%%%%%%%%%%%%%%%%%%%%%%%%%%%%%%%%%%%%%%%%%%%
%%%%%%%%%%%%%%%%%%%%%%%%%%%%%%%%%%%%%%%%%%%%%%%%%%%%%%%%%%%
\subsection{Environment parameters}
\label{section:env_parameters}

The \codeinline{env} key allows to provide the environment name, as well as optional normalization, clipping and noising of the observations:

\begin{minted}[fontsize=\scriptsize]{json}
"env": {
	"name": "CartPole-v0",
	"obs_clip": false,
	"obs_norm": true,
	"obs_noise": 0.0
}
\end{minted}

%%%%%%%%%%%%%%%%%%%%%%%%%%%%%%%%%%%%%%%%%%%%%%%%%%%%%%%%%%%
%%%%%%%%%%%%%%%%%%%%%%%%%%%%%%%%%%%%%%%%%%%%%%%%%%%%%%%%%%%
\subsection{Naming parameters}
\label{section:naming_parameters}

The \codeinline{naming} key is used to determine the way results folder will be named. This is particularly useful when multiple runs of the same environment are training using different options:

\begin{minted}[fontsize=\scriptsize]{json}
"naming": {
	"env": true,
	"agent": true,
	"tag": "test",
	"time": false,
}
\end{minted}

The \codeinline{env} and \codeinline{agent} options will respectively add the names of the environment and the agent to the folder name. The \codeinline{tag} option will add any provided string (here \codeinline{test}) to the folder name. Finally, setting \codeinline{time} to \codeinline{true} will add the launching time to the name of the folder. All these optional strings are separated by underscores.

%%%%%%%%%%%%%%%%%%%%%%%%%%%%%%%%%%%%%%%%%%%%%%%%%%%%%%%%%%%
%%%%%%%%%%%%%%%%%%%%%%%%%%%%%%%%%%%%%%%%%%%%%%%%%%%%%%%%%%%
\subsection{Trainer parameters}
\label{section:trainer_parameters}

The \codeinline{trainer} options define how the agent will be trained. Three different types of training are proposed:

\begin{enumerate}
	\item \codeinline{episode}: episode-based training for online agents. This is the regular training mode for online agents, collecting a defined amount of full episodes before training;
	\item \codeinline{buffer}: buffer-based training for online agents. This option allows to use fixed-length buffers instead of requiring full episodes. This is particularly useful when using a large amount of parallel environments, as it is not necessary to wait for the termination of all environments before launching a training. See section \textcolor{red}{to complete} for more informations;
	\item \codeinline{td}: temporal difference-based training for offline agents. This is the regular training mode of offline agents, performing by default one update after each transition.
\end{enumerate}

The specific options of each training mode are specified in section \textcolor{red}{to complete}.

%%%%%%%%%%%%%%%%%%%%%%%%%%%%%%%%%%%%%%%%%%%%%%%%%%%%%%%%%%%
%%%%%%%%%%%%%%%%%%%%%%%%%%%%%%%%%%%%%%%%%%%%%%%%%%%%%%%%%%%
\subsection{Agent parameters}
\label{section:agent_parameters}

Although the content of the \codeinline{agent} key obviously depends on the type of the selected agent, we hereafter provide an example for the \ppo agent, providing a general idea of the structure. Many examples of parameter files for the \textsc{gym} environments can be found in the \codeinline{dragonfly/env/} folder, using different agents and different training styles.

\begin{minted}[fontsize=\scriptsize]{json}
"agent": {
	"type": "ppo",
	"policy": {
		"type": "normal",
		"network": {...},
		"optimizer": {...},
		"loss": {...},
	},
	"value": {
		"type": "v_value",
		"network": {...},
		"optimizer": {...},
		"loss": {...},
	},
	"retrn": {...},
	"termination": {...}
}
\end{minted}

As can be seen, the \codeinline{type} of agent is provided first, after what, in this case, the details of the \codeinline{policy} and the \codeinline{value} structures (\textit{i.e.} the network architecture, the optimizer options as well as the loss to use) are provided. Then, the \codeinline{retrn} details, detailing the type of return vector to compute, are given. Finally, the \codeinline{termination} informations, indicating how episodes must be terminated, are provided. The specificities of all these classes are provided later in this document.