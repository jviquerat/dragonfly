%%%%%%%%%%%%%%%%%%%%%%%%%%%%%%%%%%%%%%%%%%%%%%%%%%%%%%%%%%
%%%%%%%%%%%%%%%%%%%%%%%%%%%%%%%%%%%%%%%%%%%%%%%%%%%%%%%%%%
%%%%%%%%%%%%%%%%%%%%%%%%%%%%%%%%%%%%%%%%%%%%%%%%%%%%%%%%%%
%%%%%%%%%%%%%%%%%%%%%%%%%%%%%%%%%%%%%%%%%%%%%%%%%%%%%%%%%%
\chapter{Benchmarks}

%%%%%%%%%%%%%%%%%%%%%%%%%%%%%%%%%%%%%%%%%%%%%%%%%%%%%%%%%%
%%%%%%%%%%%%%%%%%%%%%%%%%%%%%%%%%%%%%%%%%%%%%%%%%%%%%%%%%%
%%%%%%%%%%%%%%%%%%%%%%%%%%%%%%%%%%%%%%%%%%%%%%%%%%%%%%%%%%
\section{\gym benchmark (revision \codeinline{04cb588})}

The \gym library regroups several game-like environments that can be used for fast prototyping. Although the standard ones are often too easy, it is still useful to first benchmark the algorithms on them. The hyperparameters are usually slightly tuned, but not much time was dedicated to obtain the best performance on each case. Hence, these results can potentially be improved.

%%%%%%%%%%%%
%%%%%%%%%%%%
\begin{figure}
\centering
%%%%%%%%%%%%
\begin{subfigure}[t]{.35\textwidth}
	\centering
	\begin{tikzpicture}[	trim axis left, trim axis right, font=\scriptsize,
					upper/.style={	name path=upper, smooth, draw=none},
					lower/.style={	name path=lower, smooth, draw=none},]
		\begin{axis}[	xmin=0, xmax=50000, scale=0.7,
					ymin=0, ymax=220,
					scaled x ticks=false,
					xtick={0,10000,20000,30000,40000,50000},
					xticklabels={$0$,$10k$,$20k$,$30k$,$40k$,$50k$},
					ytick={0,100,200,6000},
					legend cell align=left, legend pos=south east,
					legend style={nodes={scale=0.8, transform shape}},
					every tick label/.append style={font=\scriptsize},
					grid=major, xlabel=transitions, ylabel=score]
				
			\legend{\dqn, \atc, \ppo}

			\addplot [upper, forget plot] 				table[x index=0,y index=7] {fig/gym/cartpole/dqn.dat};
			\addplot [lower, forget plot] 				table[x index=0,y index=6] {fig/gym/cartpole/dqn.dat}; 
			\addplot [fill=purple3, opacity=0.5, forget plot]	 fill between[of=upper and lower];
			\addplot[draw=purple1, thick, smooth] 		table[x index=0,y index=5] {fig/gym/cartpole/dqn.dat};
			
			\addplot [upper, forget plot] 				table[x index=0,y index=7] {fig/gym/cartpole/a2c.dat};
			\addplot [lower, forget plot] 				table[x index=0,y index=6] {fig/gym/cartpole/a2c.dat}; 
			\addplot [fill=bluegray3, opacity=0.5, forget plot]	 fill between[of=upper and lower];
			\addplot[draw=bluegray1, thick, smooth] 		table[x index=0,y index=5] {fig/gym/cartpole/a2c.dat}; 
			
 			\addplot [upper, forget plot] 				table[x index=0,y index=7] {fig/gym/cartpole/ppo.dat};
			\addplot [lower, forget plot] 				table[x index=0,y index=6] {fig/gym/cartpole/ppo.dat}; 
			\addplot [fill=gray3, opacity=0.5, forget plot] 	fill between[of=upper and lower];
			\addplot[draw=gray1, thick, smooth] 			table[x index=0,y index=5] {fig/gym/cartpole/ppo.dat}; 
			
		\end{axis}
	\end{tikzpicture}
	\caption{\codeinline{cartpole-v0}}
	\label{fig:gym_cartpole}
\end{subfigure} \qquad \qquad
%%%%%%%%%%%%
\begin{subfigure}[t]{.35\textwidth}
	\centering
	\begin{tikzpicture}[	trim axis left, trim axis right, font=\scriptsize,
					upper/.style={	name path=upper, smooth, draw=none},
					lower/.style={	name path=lower, smooth, draw=none},]
		\begin{axis}[	xmin=0, xmax=100000, scale=0.7,
					ymin=-1400, ymax=0,
					scaled x ticks=false, scaled y ticks=false,
					xtick={0,25000,50000,75000,100000},
					xticklabels={$0$,$25k$,$50k$,$75k$,$100k$},
					ytick={-1400,-1200,-1000,-800,-600,-400,-200,0},
					legend cell align=left, legend pos=south east,
					legend style={nodes={scale=0.8, transform shape}},
					every tick label/.append style={font=\scriptsize},
					grid=major, xlabel=transitions, ylabel={}]
				
			\legend{\atc, \ppo, \ddpg, \tdt, \sac}
			
			\addplot [upper, forget plot] 				table[x index=0,y index=7] {fig/gym/pendulum/a2c.dat};
			\addplot [lower, forget plot] 				table[x index=0,y index=6] {fig/gym/pendulum/a2c.dat}; 
			\addplot [fill=bluegray3, opacity=0.5, forget plot]	 fill between[of=upper and lower];
			\addplot[draw=bluegray1, thick, smooth] 		table[x index=0,y index=5] {fig/gym/pendulum/a2c.dat};
		
			\addplot [upper, forget plot] 				table[x index=0,y index=7] {fig/gym/pendulum/ppo.dat };
			\addplot [lower, forget plot] 				table[x index=0,y index=6] {fig/gym/pendulum/ppo.dat}; 
			\addplot [fill=gray3, opacity=0.5, forget plot] 	fill between[of=upper and lower];
			\addplot[draw=gray1, thick, smooth] 			table[x index=0,y index=5] {fig/gym/pendulum/ppo.dat};
			
			\addplot [upper, forget plot] 				table[x index=0,y index=7] {fig/gym/pendulum/ddpg.dat};
			\addplot [lower, forget plot] 				table[x index=0,y index=6] {fig/gym/pendulum/ddpg.dat}; 
			\addplot [fill=red3, opacity=0.5, forget plot]	 	fill between[of=upper and lower];
			\addplot[draw=red1, thick, smooth] 			table[x index=0,y index=5] {fig/gym/pendulum/ddpg.dat}; 
			
			\addplot [upper, forget plot] 				table[x index=0,y index=7] {fig/gym/pendulum/td3.dat};
			\addplot [lower, forget plot] 				table[x index=0,y index=6] {fig/gym/pendulum/td3.dat}; 
			\addplot [fill=teal3, opacity=0.5, forget plot]	fill between[of=upper and lower];
			\addplot[draw=teal1, thick, smooth] 			table[x index=0,y index=5] {fig/gym/pendulum/td3.dat}; 
			
			\addplot [upper, forget plot] 				table[x index=0,y index=7] {fig/gym/pendulum/sac.dat};
			\addplot [lower, forget plot] 				table[x index=0,y index=6] {fig/gym/pendulum/sac.dat}; 
			\addplot [fill=blue3, opacity=0.5, forget plot]	fill between[of=upper and lower];
			\addplot[draw=blue1, thick, smooth] 			table[x index=0,y index=5] {fig/gym/pendulum/sac.dat}; 
			
		\end{axis}
	\end{tikzpicture}
	\caption{\codeinline{pendulum-v1}}
	\label{fig:gym_pendulum}
\end{subfigure}
%%%%%%%%%%%%

\medskip
\medskip

%%%%%%%%%%%%
\begin{subfigure}[t]{.35\textwidth}
	\centering
	\begin{tikzpicture}[	trim axis left, trim axis right, font=\scriptsize,
					upper/.style={	name path=upper, smooth, draw=none},
					lower/.style={	name path=lower, smooth, draw=none},]
		\begin{axis}[	xmin=0, xmax=50000, scale=0.7,
					ymin=-550, ymax=-50,
					scaled x ticks=false,
					xtick={0,10000,20000,30000,40000,50000},
					xticklabels={$0$,$10k$,$20k$,$30k$,$40k$,$50k$},
					ytick={-500,-400,-300,-200,-100},
					legend cell align=left, legend pos=south east,
					legend style={nodes={scale=0.8, transform shape}},
					every tick label/.append style={font=\scriptsize},
					grid=major, xlabel=transitions, ylabel=score]
				
			\legend{\dqn, \atc, \ppo}
		
			\addplot [upper, forget plot] 				table[x index=0,y index=7] {fig/gym/acrobot/dqn.dat};
			\addplot [lower, forget plot] 				table[x index=0,y index=6] {fig/gym/acrobot/dqn.dat}; 
			\addplot [fill=purple3, opacity=0.5, forget plot]	 fill between[of=upper and lower];
			\addplot[draw=purple1, thick, smooth] 		table[x index=0,y index=5] {fig/gym/acrobot/dqn.dat}; 
			
			\addplot [upper, forget plot] 				table[x index=0,y index=7] {fig/gym/acrobot/a2c.dat};
			\addplot [lower, forget plot] 				table[x index=0,y index=6] {fig/gym/acrobot/a2c.dat}; 
			\addplot [fill=bluegray3, opacity=0.5, forget plot]	 fill between[of=upper and lower];
			\addplot[draw=bluegray1, thick, smooth] 		table[x index=0,y index=5] {fig/gym/acrobot/a2c.dat}; 
			
			\addplot [upper, forget plot] 				table[x index=0,y index=7] {fig/gym/acrobot/ppo.dat};
			\addplot [lower, forget plot] 				table[x index=0,y index=6] {fig/gym/acrobot/ppo.dat}; 
			\addplot [fill=gray3, opacity=0.5, forget plot] 	fill between[of=upper and lower];
			\addplot[draw=gray1, thick, smooth] 			table[x index=0,y index=5] {fig/gym/acrobot/ppo.dat}; 
			
		\end{axis}
	\end{tikzpicture}
	\caption{\codeinline{acrobot-v1}}
	\label{fig:gym_acrobot}
\end{subfigure} \qquad \qquad
%%%%%%%%%%%%
\begin{subfigure}[t]{.35\textwidth}
	\centering
	\begin{tikzpicture}[	trim axis left, trim axis right, font=\scriptsize,
					upper/.style={	name path=upper, smooth, draw=none},
					lower/.style={	name path=lower, smooth, draw=none},]
		\begin{axis}[	xmin=0, xmax=500000, scale=0.7,
					ymin=-350, ymax=350,
					scaled x ticks=false,
					xtick={0,250000,500000,750000,1000000},
					xticklabels={$0$,$250k$,$500k$,$750k$,$1000k$},
					ytick={-300,-200,-100,0,100,200,300},
					legend cell align=left, legend pos=south east,
					legend style={nodes={scale=0.8, transform shape}},
					every tick label/.append style={font=\scriptsize},
					grid=major, xlabel=transitions, ylabel=score]
				
			\legend{\ppo, \ddpg, \tdt, \sac}
			
			\addplot [upper, forget plot] 				table[x index=0,y index=7] {fig/gym/lunarlander/ppo.dat};
			\addplot [lower, forget plot] 				table[x index=0,y index=6] {fig/gym/lunarlander/ppo.dat}; 
			\addplot [fill=gray3, opacity=0.5, forget plot] 	fill between[of=upper and lower];
			\addplot[draw=gray1, thick, smooth] 			table[x index=0,y index=5] {fig/gym/lunarlander/ppo.dat}; 
			
			\addplot [upper, forget plot] 				table[x index=0,y index=7] {fig/gym/lunarlander/ddpg.dat};
			\addplot [lower, forget plot] 				table[x index=0,y index=6] {fig/gym/lunarlander/ddpg.dat}; 
			\addplot [fill=red3, opacity=0.5, forget plot]	 	fill between[of=upper and lower];
			\addplot[draw=red1, thick, smooth] 			table[x index=0,y index=5] {fig/gym/lunarlander/ddpg.dat}; 
			
			\addplot [upper, forget plot] 				table[x index=0,y index=7] {fig/gym/lunarlander/td3.dat};
			\addplot [lower, forget plot] 				table[x index=0,y index=6] {fig/gym/lunarlander/td3.dat}; 
			\addplot [fill=teal3, opacity=0.5, forget plot]	fill between[of=upper and lower];
			\addplot[draw=teal1, thick, smooth] 			table[x index=0,y index=5] {fig/gym/lunarlander/td3.dat}; 
			
			\addplot [upper, forget plot] 				table[x index=0,y index=7] {fig/gym/lunarlander/sac.dat};
			\addplot [lower, forget plot] 				table[x index=0,y index=6] {fig/gym/lunarlander/sac.dat}; 
			\addplot [fill=blue3, opacity=0.5, forget plot]	fill between[of=upper and lower];
			\addplot[draw=blue1, thick, smooth] 			table[x index=0,y index=5] {fig/gym/lunarlander/sac.dat}; 
			
		\end{axis}
	\end{tikzpicture}
	\caption{\codeinline{lunarlandercontinuous-v2}}
	\label{fig:gym_lunarlander}
\end{subfigure}
%%%%%%%%%%%%

\medskip
\medskip

%%%%%%%%%%%%
\begin{subfigure}[t]{.35\textwidth}
	\centering
	\begin{tikzpicture}[	trim axis left, trim axis right, font=\scriptsize,
					upper/.style={	name path=upper, smooth, draw=none},
					lower/.style={	name path=lower, smooth, draw=none},]
		\begin{axis}[	xmin=0, xmax=200000, scale=0.7,
					ymin=-220, ymax=-100,
					scaled x ticks=false, scaled y ticks=false,
					xtick={0,50000,100000,150000,200000},
					xticklabels={$0$,$50k$,$100k$,$150k$,$200k$},
					ytick={-200,-150,-100},
					legend cell align=left, legend pos=south east,
					legend style={nodes={scale=0.8, transform shape}},
					every tick label/.append style={font=\scriptsize},
					grid=major, xlabel=transitions, ylabel={}]
				
			\legend{\dqn, \atc, \ppo}

			\addplot [upper, forget plot] 				table[x index=0,y index=7] {fig/gym/mountaincar/dqn.dat};
			\addplot [lower, forget plot] 				table[x index=0,y index=6] {fig/gym/mountaincar/dqn.dat}; 
			\addplot [fill=purple3, opacity=0.5, forget plot]	 fill between[of=upper and lower];
			\addplot[draw=purple1, thick, smooth] 		table[x index=0,y index=5] {fig/gym/mountaincar/dqn.dat}; 

			\addplot [upper, forget plot] 				table[x index=0,y index=7] {fig/gym/mountaincar/a2c.dat};
			\addplot [lower, forget plot] 				table[x index=0,y index=6] {fig/gym/mountaincar/a2c.dat}; 
			\addplot [fill=bluegray3, opacity=0.5, forget plot]	 fill between[of=upper and lower];
			\addplot[draw=bluegray1, thick, smooth] 		table[x index=0,y index=5] {fig/gym/mountaincar/a2c.dat}; 

			\addplot [upper, forget plot] 				table[x index=0,y index=7] {fig/gym/mountaincar/ppo.dat};
			\addplot [lower, forget plot] 				table[x index=0,y index=6] {fig/gym/mountaincar/ppo.dat}; 
			\addplot [fill=gray3, opacity=0.5, forget plot] 	fill between[of=upper and lower];
			\addplot[draw=gray1, thick, smooth] 			table[x index=0,y index=5] {fig/gym/mountaincar/ppo.dat}; 
			
		\end{axis}
	\end{tikzpicture}
	\caption{\codeinline{mountaincar-v0}}
	\label{fig:gym_mountaincar}
\end{subfigure} \qquad \qquad
%%%%%%%%%%%%
\begin{subfigure}[t]{.35\textwidth}
	\centering
	\begin{tikzpicture}[	trim axis left, trim axis right, font=\scriptsize,
					upper/.style={	name path=upper, smooth, draw=none},
					lower/.style={	name path=lower, smooth, draw=none},]
		\begin{axis}[	xmin=0, xmax=1000000, scale=0.7,
					ymin=-150, ymax=350,
					scaled x ticks=false,
					xtick={0,250000,500000,750000,1000000},
					xticklabels={$0$,$250k$,$500k$,$750k$,$1000k$},
					ytick={-100,0,100,200,300},
					legend cell align=left, legend pos=south east,
					legend style={nodes={scale=0.8, transform shape}},
					every tick label/.append style={font=\scriptsize},
					grid=major, xlabel=transitions, ylabel={}]
				
			\legend{\ppo, \ddpg, \tdt, \sac} 

			\addplot [upper, forget plot] 				table[x index=0,y index=7] {fig/gym/bipedalwalker/ppo.dat };
			\addplot [lower, forget plot] 				table[x index=0,y index=6] {fig/gym/bipedalwalker/ppo.dat}; 
			\addplot [fill=gray3, opacity=0.5, forget plot] 	fill between[of=upper and lower];
			\addplot[draw=gray1, thick, smooth] 			table[x index=0,y index=5] {fig/gym/bipedalwalker/ppo.dat}; 
			
			\addplot [upper, forget plot] 				table[x index=0,y index=7] {fig/gym/bipedalwalker/ddpg.dat};
			\addplot [lower, forget plot] 				table[x index=0,y index=6] {fig/gym/bipedalwalker/ddpg.dat}; 
			\addplot [fill=red3, opacity=0.5, forget plot]	 	fill between[of=upper and lower];
			\addplot[draw=red1, thick, smooth] 			table[x index=0,y index=5] {fig/gym/bipedalwalker/ddpg.dat}; 
			
			\addplot [upper, forget plot] 				table[x index=0,y index=7] {fig/gym/bipedalwalker/td3.dat};
			\addplot [lower, forget plot] 				table[x index=0,y index=6] {fig/gym/bipedalwalker/td3.dat}; 
			\addplot [fill=teal3, opacity=0.5, forget plot]	fill between[of=upper and lower];
			\addplot[draw=teal1, thick, smooth] 			table[x index=0,y index=5] {fig/gym/bipedalwalker/td3.dat}; 
			
			\addplot [upper, forget plot] 				table[x index=0,y index=7] {fig/gym/bipedalwalker/sac.dat};
			\addplot [lower, forget plot] 				table[x index=0,y index=6] {fig/gym/bipedalwalker/sac.dat}; 
			\addplot [fill=blue3, opacity=0.5, forget plot]	fill between[of=upper and lower];
			\addplot[draw=blue1, thick, smooth] 			table[x index=0,y index=5] {fig/gym/bipedalwalker/sac.dat}; 
			
		\end{axis}
	\end{tikzpicture}
	\caption{\codeinline{bipedalwalker-v3}}
	\label{fig:gym_bipedal}
\end{subfigure}
%%%%%%%%%%%%
\caption{\textbf{Performance comparison on the \gym benchmark.}} 
\label{fig:gym}
\end{figure} 
%%%%%%%%%%%%
%%%%%%%%%%%%

%%%%%%%%%%%%%%%%%%%%%%%%%%%%%%%%%%%%%%%%%%%%%%%%%%%%%%%%%%
%%%%%%%%%%%%%%%%%%%%%%%%%%%%%%%%%%%%%%%%%%%%%%%%%%%%%%%%%%
%%%%%%%%%%%%%%%%%%%%%%%%%%%%%%%%%%%%%%%%%%%%%%%%%%%%%%%%%%
\section{\mujoco locomotion benchmark (revision \codeinline{04cb588})}

The \mujoco locomotion benchmark consists of several locomotion problems where humanoids and other sorts of multipedal beasts learn to walk \cite{mujoco}. \warning{The goal here is not to obtain the best performance on this benchmark}, but only to check that the implementations (mostly \tdt and \sac) are in line with the OpenAI benchmark results \url{https://spinningup.openai.com/en/latest/spinningup/bench.html}. For this reason, the same set of hyperparameters is used. Exceptions are:

\begin{enumerate}
	\item The standard deviations for \ppo are state-based;
	\item The standard deviations for \sac are not generated through their log value;
	\item \codeinline{swimmer-v4} is known to be difficult to solve. \ppo can solve it with $\gamma = 0.9995$ instead of the standard $\gamma = 0.99$ (see \url{https://rl-vs.github.io/rlvs2021/pg-pitfalls.html} at around 45 minutes of the video).
\end{enumerate}

%%%%%%%%%%%%
%%%%%%%%%%%%
\begin{figure}
\centering
%%%%%%%%%%%%
\begin{subfigure}[t]{.35\textwidth}
	\centering
	\begin{tikzpicture}[	trim axis left, trim axis right, font=\scriptsize,
					upper/.style={	name path=upper, smooth, draw=none},
					lower/.style={	name path=lower, smooth, draw=none},]
		\begin{axis}[	xmin=0, xmax=1000000, scale=0.7,
					ymin=-1000, ymax=6000,
					scaled x ticks=false,
					xtick={0,250000,500000,750000,1000000},
					xticklabels={$0$,$250k$,$500k$,$750k$,$1000k$},
					ytick={-1000,0,2000,4000,6000},
					yticklabels={$-1k$,$0$,$2k$,$4k$,$6k$},
					legend cell align=left, legend pos=south east,
					legend style={nodes={scale=0.8, transform shape}},
					every tick label/.append style={font=\scriptsize},
					grid=major, xlabel=transitions, ylabel=score]
				
			\legend{\ppo, \ddpg, \tdt, \sac}
			
			\addplot [upper, forget plot] 				table[x index=0,y index=7] {fig/mujoco/ant-v4_ppo};
			\addplot [lower, forget plot] 				table[x index=0,y index=6] {fig/mujoco/ant-v4_ppo}; 
			\addplot [fill=gray3, opacity=0.5, forget plot] 	fill between[of=upper and lower];
			\addplot[draw=gray1, thick, smooth] 			table[x index=0,y index=5] {fig/mujoco/ant-v4_ppo};
			
			\addplot [upper, forget plot] 				table[x index=0,y index=7] {fig/mujoco/ant-v4_ddpg};
			\addplot [lower, forget plot] 				table[x index=0,y index=6] {fig/mujoco/ant-v4_ddpg}; 
			\addplot [fill=red3, opacity=0.5, forget plot] 	fill between[of=upper and lower];
			\addplot[draw=red1, thick, smooth] 			table[x index=0,y index=5] {fig/mujoco/ant-v4_ddpg}; 
			
			\addplot [upper, forget plot] 				table[x index=0,y index=7] {fig/mujoco/ant-v4_td3};
			\addplot [lower, forget plot] 				table[x index=0,y index=6] {fig/mujoco/ant-v4_td3}; 
			\addplot [fill=teal3, opacity=0.5, forget plot] 	fill between[of=upper and lower];
			\addplot[draw=teal1, thick, smooth] 			table[x index=0,y index=5] {fig/mujoco/ant-v4_td3}; 
		
			\addplot [upper, forget plot] 				table[x index=0,y index=7] {fig/mujoco/ant-v4_sac};
			\addplot [lower, forget plot] 				table[x index=0,y index=6] {fig/mujoco/ant-v4_sac}; 
			\addplot [fill=blue3, opacity=0.5, forget plot] 	fill between[of=upper and lower];
			\addplot[draw=blue1, thick, smooth] 			table[x index=0,y index=5] {fig/mujoco/ant-v4_sac};
			
		\end{axis}
	\end{tikzpicture}
	\caption{\codeinline{ant-v4}}
	\label{fig:mujoco_ant}
\end{subfigure} \qquad \qquad
%%%%%%%%%%%%
\begin{subfigure}[t]{.35\textwidth}
	\centering
	\begin{tikzpicture}[	trim axis left, trim axis right, font=\scriptsize,
					upper/.style={	name path=upper, smooth, draw=none},
					lower/.style={	name path=lower, smooth, draw=none},]
		\begin{axis}[	xmin=0, xmax=1000000, scale=0.7,
					ymin=-500, ymax=12000,
					scaled x ticks=false, scaled y ticks=false,
					xtick={0,250000,500000,750000,1000000},
					xticklabels={$0$,$250k$,$500k$,$750k$,$1000k$},
					ytick={0,2000,4000,6000,8000,10000,12000},
					yticklabels={$0$,$2k$,$4k$,$6k$,$8k$,$10k$,$12k$},
					legend cell align=left, legend pos=south east,
					legend style={nodes={scale=0.8, transform shape}},
					every tick label/.append style={font=\scriptsize},
					grid=major, xlabel=transitions, ylabel={}]
				
			\legend{\ppo, \ddpg, \tdt, \sac}
			
			\addplot [upper, forget plot] 				table[x index=0,y index=7] {fig/mujoco/halfcheetah-v4_ppo};
			\addplot [lower, forget plot] 				table[x index=0,y index=6] {fig/mujoco/halfcheetah-v4_ppo}; 
			\addplot [fill=gray3, opacity=0.5, forget plot] 	fill between[of=upper and lower];
			\addplot[draw=gray1, thick, smooth] 			table[x index=0,y index=5] {fig/mujoco/halfcheetah-v4_ppo};
			
			\addplot [upper, forget plot] 				table[x index=0,y index=7] {fig/mujoco/halfcheetah-v4_ddpg};
			\addplot [lower, forget plot] 				table[x index=0,y index=6] {fig/mujoco/halfcheetah-v4_ddpg}; 
			\addplot [fill=red3, opacity=0.5, forget plot] 	fill between[of=upper and lower];
			\addplot[draw=red1, thick, smooth] 			table[x index=0,y index=5] {fig/mujoco/halfcheetah-v4_ddpg}; 
			
			\addplot [upper, forget plot] 				table[x index=0,y index=7] {fig/mujoco/halfcheetah-v4_td3};
			\addplot [lower, forget plot] 				table[x index=0,y index=6] {fig/mujoco/halfcheetah-v4_td3}; 
			\addplot [fill=teal3, opacity=0.5, forget plot]	fill between[of=upper and lower];
			\addplot[draw=teal1, thick, smooth] 			table[x index=0,y index=5] {fig/mujoco/halfcheetah-v4_td3};  
		
			\addplot [upper, forget plot] 				table[x index=0,y index=7] {fig/mujoco/halfcheetah-v4_sac};
			\addplot [lower, forget plot] 				table[x index=0,y index=6] {fig/mujoco/halfcheetah-v4_sac}; 
			\addplot [fill=blue3, opacity=0.5, forget plot] 	fill between[of=upper and lower];
			\addplot[draw=blue1, thick, smooth] 			table[x index=0,y index=5] {fig/mujoco/halfcheetah-v4_sac};
			
		\end{axis}
	\end{tikzpicture}
	\caption{\codeinline{halfcheetah-v4}}
	\label{fig:mujoco_halfcheetah}
\end{subfigure}
%%%%%%%%%%%%

\medskip
\medskip

%%%%%%%%%%%%
\begin{subfigure}[t]{.35\textwidth}
	\centering
	\begin{tikzpicture}[	trim axis left, trim axis right, font=\scriptsize,
					upper/.style={	name path=upper, smooth, draw=none},
					lower/.style={	name path=lower, smooth, draw=none},]
		\begin{axis}[	xmin=0, xmax=1000000, scale=0.7,
					ymin=0, ymax=4000,
					scaled x ticks=false,
					xtick={0,250000,500000,750000,1000000},
					xticklabels={$0$,$250k$,$500k$,$750k$,$1000k$},
					ytick={0,1000,2000,3000,4000},
					yticklabels={$0$,$1k$,$2k$,$3k$,$4k$},
					legend cell align=left, legend pos=south east,
					legend style={nodes={scale=0.8, transform shape}},
					every tick label/.append style={font=\scriptsize},
					grid=major, xlabel=transitions, ylabel=score]
				
			\legend{\ppo, \ddpg, \tdt, \sac}
			
			\addplot [upper, forget plot] 				table[x index=0,y index=7] {fig/mujoco/hopper-v4_ppo};
			\addplot [lower, forget plot] 				table[x index=0,y index=6] {fig/mujoco/hopper-v4_ppo}; 
			\addplot [fill=gray3, opacity=0.5, forget plot] 	fill between[of=upper and lower];
			\addplot[draw=gray1, thick, smooth] 			table[x index=0,y index=5] {fig/mujoco/hopper-v4_ppo};
			
			\addplot [upper, forget plot] 				table[x index=0,y index=7] {fig/mujoco/hopper-v4_ddpg};
			\addplot [lower, forget plot] 				table[x index=0,y index=6] {fig/mujoco/hopper-v4_ddpg}; 
			\addplot [fill=red3, opacity=0.5, forget plot] 	fill between[of=upper and lower];
			\addplot[draw=red1, thick, smooth] 			table[x index=0,y index=5] {fig/mujoco/hopper-v4_ddpg}; 
			
			\addplot [upper, forget plot] 				table[x index=0,y index=7] {fig/mujoco/hopper-v4_td3};
			\addplot [lower, forget plot] 				table[x index=0,y index=6] {fig/mujoco/hopper-v4_td3}; 
			\addplot [fill=teal3, opacity=0.5, forget plot]	fill between[of=upper and lower];
			\addplot[draw=teal1, thick, smooth] 			table[x index=0,y index=5] {fig/mujoco/hopper-v4_td3}; 
		
			\addplot [upper, forget plot] 				table[x index=0,y index=7] {fig/mujoco/hopper-v4_sac};
			\addplot [lower, forget plot] 				table[x index=0,y index=6] {fig/mujoco/hopper-v4_sac}; 
			\addplot [fill=blue3, opacity=0.5, forget plot] 	fill between[of=upper and lower];
			\addplot[draw=blue1, thick, smooth] 			table[x index=0,y index=5] {fig/mujoco/hopper-v4_sac};
			
		\end{axis}
	\end{tikzpicture}
	\caption{\codeinline{hopper-v4}}
	\label{fig:mujoco_hopper}
\end{subfigure} \qquad \qquad
%%%%%%%%%%%%
\begin{subfigure}[t]{.35\textwidth}
	\centering
	\begin{tikzpicture}[	trim axis left, trim axis right, font=\scriptsize,
					upper/.style={	name path=upper, smooth, draw=none},
					lower/.style={	name path=lower, smooth, draw=none},]
		\begin{axis}[	xmin=0, xmax=1000000, scale=0.7,
					ymin=0, ymax=6000,
					scaled x ticks=false, scaled y ticks=false,
					xtick={0,250000,500000,750000,1000000},
					xticklabels={$0$,$250k$,$500k$,$750k$,$1000k$},
					ytick={0,2000,4000,6000},
					yticklabels={$0$,$2k$,$4k$,$6k$},
					legend cell align=left, legend pos=south east,
					legend style={nodes={scale=0.8, transform shape}},
					every tick label/.append style={font=\scriptsize},
					grid=major, xlabel=transitions, ylabel={}]
				
			\legend{\ppo, \tdt, \sac}
			
			\addplot [upper, forget plot] 				table[x index=0,y index=7] {fig/mujoco/humanoid/ppo.dat };
			\addplot [lower, forget plot] 				table[x index=0,y index=6] {fig/mujoco/humanoid/ppo.dat}; 
			\addplot [fill=gray3, opacity=0.5, forget plot] 	fill between[of=upper and lower];
			\addplot[draw=gray1, thick, smooth] 			table[x index=0,y index=5] {fig/mujoco/humanoid/ppo.dat}; 
			
			\addplot [upper, forget plot] 				table[x index=0,y index=7] {fig/mujoco/humanoid/td3.dat};
			\addplot [lower, forget plot] 				table[x index=0,y index=6] {fig/mujoco/humanoid/td3.dat}; 
			\addplot [fill=teal3, opacity=0.5, forget plot]	fill between[of=upper and lower];
			\addplot[draw=teal1, thick, smooth] 			table[x index=0,y index=5] {fig/mujoco/humanoid/td3.dat}; 
		
			\addplot [upper, forget plot] 				table[x index=0,y index=7] {fig/mujoco/humanoid/sac.dat};
			\addplot [lower, forget plot] 				table[x index=0,y index=6] {fig/mujoco/humanoid/sac.dat}; 
			\addplot [fill=blue3, opacity=0.5, forget plot] 	fill between[of=upper and lower];
			\addplot[draw=blue1, thick, smooth] 			table[x index=0,y index=5] {fig/mujoco/humanoid/sac.dat};
			
		\end{axis}
	\end{tikzpicture}
	\caption{\codeinline{humanoid-v4}}
	\label{fig:mujoco_humanoid}
\end{subfigure}
%%%%%%%%%%%%

\medskip
\medskip

%%%%%%%%%%%%
\begin{subfigure}[t]{.35\textwidth}
	\centering
	\begin{tikzpicture}[	trim axis left, trim axis right, font=\scriptsize,
					upper/.style={	name path=upper, smooth, draw=none},
					lower/.style={	name path=lower, smooth, draw=none},]
		\begin{axis}[	xmin=0, xmax=1000000, scale=0.7,
					ymin=0, ymax=5000,
					scaled x ticks=false,
					xtick={0,250000,500000,750000,1000000},
					xticklabels={$0$,$250k$,$500k$,$750k$,$1000k$},
					ytick={0,1000,2000,3000,4000,5000},
					yticklabels={$0$,$1k$,$2k$,$3k$,$4k$,$5k$},
					legend cell align=left, legend pos=south east,
					legend style={nodes={scale=0.8, transform shape}},
					every tick label/.append style={font=\scriptsize},
					grid=major, xlabel=transitions, ylabel=score]
				
			\legend{\ppo, \ddpg, \tdt, \sac}
			
			\addplot [upper, forget plot] 				table[x index=0,y index=7] {fig/mujoco/walker2d-v4_ppo};
			\addplot [lower, forget plot] 				table[x index=0,y index=6] {fig/mujoco/walker2d-v4_ppo}; 
			\addplot [fill=gray3, opacity=0.5, forget plot] 	fill between[of=upper and lower];
			\addplot[draw=gray1, thick, smooth] 			table[x index=0,y index=5] {fig/mujoco/walker2d-v4_ppo};

			\addplot [upper, forget plot] 				table[x index=0,y index=7] {fig/mujoco/swimmer-v4_ddpg};
			\addplot [lower, forget plot] 				table[x index=0,y index=6] {fig/mujoco/swimmer-v4_ddpg}; 
			\addplot [fill=red3, opacity=0.5, forget plot] 	fill between[of=upper and lower];
			\addplot[draw=red1, thick, smooth] 			table[x index=0,y index=5] {fig/mujoco/swimmer-v4_ddpg}; 
			
			\addplot [upper, forget plot] 				table[x index=0,y index=7] {fig/mujoco/walker2d-v4_td3};
			\addplot [lower, forget plot] 				table[x index=0,y index=6] {fig/mujoco/walker2d-v4_td3}; 
			\addplot [fill=teal3, opacity=0.5, forget plot]	fill between[of=upper and lower];
			\addplot[draw=teal1, thick, smooth] 			table[x index=0,y index=5] {fig/mujoco/walker2d-v4_td3}; 
		
			\addplot [upper, forget plot] 				table[x index=0,y index=7] {fig/mujoco/walker2d-v4_sac};
			\addplot [lower, forget plot] 				table[x index=0,y index=6] {fig/mujoco/walker2d-v4_sac}; 
			\addplot [fill=blue3, opacity=0.5, forget plot] 	fill between[of=upper and lower];
			\addplot[draw=blue1, thick, smooth] 			table[x index=0,y index=5] {fig/mujoco/walker2d-v4_sac};
			
		\end{axis}
	\end{tikzpicture}
	\caption{\codeinline{walker2d-v4}}
	\label{fig:mujoco_walker2d}
\end{subfigure} \qquad \qquad
%%%%%%%%%%%%
\begin{subfigure}[t]{.35\textwidth}
	\centering
	\begin{tikzpicture}[	trim axis left, trim axis right, font=\scriptsize,
					upper/.style={	name path=upper, smooth, draw=none},
					lower/.style={	name path=lower, smooth, draw=none},]
		\begin{axis}[	xmin=0, xmax=1000000, scale=0.7,
					ymin=0, ymax=300,
					scaled x ticks=false, scaled y ticks=false,
					xtick={0,250000,500000,750000,1000000},
					xticklabels={$0$,$250k$,$500k$,$750k$,$1000k$},
					ytick={0,100,200,300},
					legend cell align=left, legend pos=south east,
					legend style={nodes={scale=0.8, transform shape}},
					every tick label/.append style={font=\scriptsize},
					grid=major, xlabel=transitions, ylabel={}]
				
			\legend{\ppo, \ddpg, \tdt, \sac}
			
			\addplot [upper, forget plot] 				table[x index=0,y index=7] {fig/mujoco/swimmer-v4_ppo};
			\addplot [lower, forget plot] 				table[x index=0,y index=6] {fig/mujoco/swimmer-v4_ppo}; 
			\addplot [fill=gray3, opacity=0.5, forget plot] 	fill between[of=upper and lower];
			\addplot[draw=gray1, thick, smooth] 			table[x index=0,y index=5] {fig/mujoco/swimmer-v4_ppo};
			
			\addplot [upper, forget plot] 				table[x index=0,y index=7] {fig/mujoco/swimmer-v4_ddpg};
			\addplot [lower, forget plot] 				table[x index=0,y index=6] {fig/mujoco/swimmer-v4_ddpg}; 
			\addplot [fill=red3, opacity=0.5, forget plot] 	fill between[of=upper and lower];
			\addplot[draw=red1, thick, smooth] 			table[x index=0,y index=5] {fig/mujoco/swimmer-v4_ddpg}; 
			
			\addplot [upper, forget plot] 				table[x index=0,y index=7] {fig/mujoco/swimmer-v4_td3};
			\addplot [lower, forget plot] 				table[x index=0,y index=6] {fig/mujoco/swimmer-v4_td3}; 
			\addplot [fill=teal3, opacity=0.5, forget plot]	fill between[of=upper and lower];
			\addplot[draw=teal1, thick, smooth] 			table[x index=0,y index=5] {fig/mujoco/swimmer-v4_td3}; 
		
			\addplot [upper, forget plot] 				table[x index=0,y index=7] {fig/mujoco/swimmer-v4_sac};
			\addplot [lower, forget plot] 				table[x index=0,y index=6] {fig/mujoco/swimmer-v4_sac}; 
			\addplot [fill=blue3, opacity=0.5, forget plot] 	fill between[of=upper and lower];
			\addplot[draw=blue1, thick, smooth] 			table[x index=0,y index=5] {fig/mujoco/swimmer-v4_sac};
			
		\end{axis}
	\end{tikzpicture}
	\caption{\codeinline{swimmer-v4}}
	\label{fig:mujoco_swimmer}
\end{subfigure}
%%%%%%%%%%%%
\caption{\textbf{Performance comparison on the \mujoco benchmark.}} 
\label{fig:mujoco}
\end{figure} 
%%%%%%%%%%%%
%%%%%%%%%%%%